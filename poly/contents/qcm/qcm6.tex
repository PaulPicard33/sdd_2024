%-*- coding: iso-latin-1 -*-
\section{QCM}
\paragraph{Question 1.} Quand le nombre d'observations tend vers l'infini, 
\begin{itemize}
\item[$\square$] le risque empirique d'un mod�le converge vers le risque de ce mod�le ; 
\item[$\square$] le risque empirique minimal converge vers le risque minimal ; 
\item[$\square$] le minimiseur du risque empirique converge vers le minimiseur du risque.
\end{itemize}

\paragraph{Question 2.} Supposons un probl�me de classification en 2
dimensions, avec $n$ observations. Nous consid�rons comme espace des hypoth�ses
l'ensemble des unions de $K$ cercles ($K > 0$ est fix�) : les points int�rieurs
� ces cercles sont �tiquet�s positifs, les autres n�gatifs. Alors
\begin{itemize}
\item[$\square$] Il ne s'agit pas d'un mod�le param�trique.
\item[$\square$] Il s'agit d'un mod�le param�trique � $K$ param�tres.
\item[$\square$] Il s'agit d'un mod�le param�trique � $2K$ param�tres.
\item[$\square$] Il s'agit d'un mod�le param�trique � $3K$ param�tres.
\end{itemize}

\paragraph{Question 3.} Quel algorithme pr�f�rer pour entra�ner une r�gression
lin�aire sur un jeu de donn�es contenant $n$ observations et $p$ variables :
\begin{itemize}
\item Si $n=10^5$ et $p=5$ ?
  \begin{itemize}
  \item[$\square$] Une inversion de matrice.
  \item[$\square$] Un algorithme du gradient.
  \end{itemize}
\item Si $n=10^5$ et $p=10^5$ ?
  \begin{itemize}
  \item[$\square$] Une inversion de matrice.
  \item[$\square$] Un algorithme du gradient.
  \end{itemize}
\end{itemize}

\section*{Solution}
{%
\noindent
\rotatebox[origin=c]{180}{%
\noindent
\begin{minipage}[t]{\linewidth}
\paragraph{Question 1.} Seule la premi�re proposition est vraie. \newline

\paragraph{Question 2.} Il s'agit d'un mod�le param�trique et nous avons besoin
de $3K$ param�tres pour d�terminer les coordonn�es de $K$ cercles (coordonn�es
du centre + rayon). \newline

\paragraph{Question 3.}  Lorsque la matrice $X^\top X$ (de dimensions
$p \times p$) est de petite taille (peu de variables), on pourra utiliser un
algorithme d'inversion de matrice. Sinon, un algorithme du gradient sera plus
appropri�.
\end{minipage}%
}%

%%% Local Variables:
%%% mode: latex
%%% TeX-master: "../../sdd_2024_poly"
%%% End:
